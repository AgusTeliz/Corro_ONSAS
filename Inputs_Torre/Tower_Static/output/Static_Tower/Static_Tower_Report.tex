\documentclass[a4paper,10pt]{article} 
\usepackage[a4paper,margin=20mm]{geometry} 
\usepackage{longtable} 
\usepackage{float} 
\usepackage{adjustbox} 
\usepackage{graphicx} 
\usepackage{color} 
\usepackage{booktabs} 
\aboverulesep=0ex 
\belowrulesep=0ex 
\usepackage{array} 
\newcolumntype{?}{!{\vrule width 1pt}}
\usepackage{fancyhdr} 
\pagestyle{fancy} 
\fancyhf{} 
\lhead{Report ONSAS version 0.1.10} 
\rhead{Problem: StaticTower } 
\lfoot{Date: \today} 
\rfoot{Page \thepage} 
\renewcommand{\footrulewidth}{1pt} 
\renewcommand{\headrulewidth}{1.5pt} 
\setlength{\parindent}{0pt} 
\usepackage[T1]{fontenc} 
\usepackage{libertine} 
\usepackage{arydshln} 
\definecolor{miblue}{rgb}{0,0.1,0.38} 
\usepackage{titlesec} 
\titleformat{\section}{\normalfont\Large\color{miblue}\bfseries}{\color{miblue}\sectionmark\thesection}{0.5em}{}[{\color{miblue}\titlerule[0.5pt]}] 

\begin{document} 
\begin{center} 
\textbf{ONSAS v.0.1.10 analysis report \\ Problem: StaticTower} 
\end{center} 

This is an ONSAS automatically-generated report with part of the results obtained after the analysis. The user can access other magnitudes and results through the GNU-Octave/MATLAB console. The code is provided AS IS \textbf{WITHOUT WARRANTY of any kind}, express or implied.

\section{Analysis results}

\begin{longtable}{cccccccc} 
$\#t$ & $ \lambda(t)$ & its & $\| RHS \|$ & $\| \Delta u \|$ & flagExit  & npos & nneg  \\ \hline 
 \endhead 
\hdashline
   1 &  0.00e+00 &    0 &           &           &  0 &   0 &   0 \\ 
     &           &    1 &  3.35e+02 &  9.72e-02 &    &     &     \\ 
     &           &    2 &  8.42e-04 &  1.17e-04 &    &     &     \\ 
     &           &    3 &  1.75e-05 &  2.77e-09 &    &     &     \\ 
\hdashline
   2 &  1.00e+02 &    3 &           &           &  1 &  10 &   0 \\ 
     &           &    1 &  2.76e+02 &  9.66e-02 &    &     &     \\ 
     &           &    2 &  3.43e-05 &  1.66e-05 &    &     &     \\ 
\hdashline
   3 &  2.00e+02 &    2 &           &           &  1 &  10 &   0 \\ 
     &           &    1 &  2.76e+02 &  9.66e-02 &    &     &     \\ 
     &           &    2 &  3.23e-05 &  1.66e-05 &    &     &     \\ 
\hdashline
   4 &  3.00e+02 &    2 &           &           &  1 &  10 &   0 \\ 
     &           &    1 &  2.76e+02 &  9.66e-02 &    &     &     \\ 
     &           &    2 &  3.49e-05 &  1.66e-05 &    &     &     \\ 
\hdashline
   5 &  4.00e+02 &    2 &           &           &  1 &  10 &   0 \\ 
     &           &    1 &  2.76e+02 &  9.66e-02 &    &     &     \\ 
     &           &    2 &  3.41e-05 &  1.66e-05 &    &     &     \\ 
\hdashline
   6 &  5.00e+02 &    2 &           &           &  1 &  10 &   0 \\ 
     &           &    1 &  2.76e+02 &  9.66e-02 &    &     &     \\ 
     &           &    2 &  3.39e-05 &  1.66e-05 &    &     &     \\ 
\hdashline
   7 &  6.00e+02 &    2 &           &           &  1 &  10 &   0 \\ 
     &           &    1 &  2.76e+02 &  9.66e-02 &    &     &     \\ 
     &           &    2 &  3.41e-05 &  1.66e-05 &    &     &     \\ 
\hdashline
   8 &  7.00e+02 &    2 &           &           &  1 &  10 &   0 \\ 
     &           &    1 &  2.76e+02 &  9.66e-02 &    &     &     \\ 
     &           &    2 &  3.47e-05 &  1.66e-05 &    &     &     \\ 
\hdashline
   9 &  8.00e+02 &    2 &           &           &  1 &  10 &   0 \\ 
     &           &    1 &  2.76e+02 &  9.66e-02 &    &     &     \\ 
     &           &    2 &  3.41e-05 &  1.66e-05 &    &     &     \\ 
\hdashline
  10 &  9.00e+02 &    2 &           &           &  1 &  10 &   0 \\ 
     &           &    1 &  2.76e+02 &  9.66e-02 &    &     &     \\ 
     &           &    2 &  3.46e-05 &  1.66e-05 &    &     &     \\ 
\hdashline
  11 &  1.00e+03 &    2 &           &           &  1 &  10 &   0 \\ 
 
\caption{Output of incremental analysis.}
\end{longtable}

\newpage 

\end{document}